\documentclass{aa} 

\usepackage{graphicx,natbib,url,twoopt}
\usepackage[varg]{txfonts}           %% A&A font choice
\usepackage{hyperref}                %% for pdflatex
%%\usepackage[breaklinks]{hyperref}  %% for latex+dvips
%%\usepackage{breakurl}              %% for latex+dvips
\usepackage{pdfcomment}              %% for popup acronym meanings
\usepackage{acronym}                 %% for popup acronym meanings

\hypersetup{
  colorlinks=true,   %% links colored instead of frames
  urlcolor=blue,     %% external hyperlinks
  linkcolor=red,     %% internal latex links (eg Fig)
}

\bibpunct{(}{)}{;}{a}{}{,}    %% natbib cite format used by A&A and ApJ

\pagestyle{plain}   %% undo the fancy A&A pagestyle 

%% Add commands to add a note or link to a reference
\makeatletter
\newcommand{\bibnote}[2]{\@namedef{#1note}{#2}}
\newcommand{\biblink}[2]{\@namedef{#1link}{#2}}
\makeatother

\makeatletter
 \newcommandtwoopt{\citeads}[3][][]{%
   \nonstopmode%              %% fix to not stop at error message in latex
   \href{http://adsabs.harvard.edu/abs/#3}%
        {\def\hyper@linkstart##1##2{}%
         \let\hyper@linkend\@empty\citealp[#1][#2]{#3}}%   %% Rutten, 2000
   \biblink{#3}{\href{http://adsabs.harvard.edu/abs/#3}{ADS}}%
   \errorstopmode}            %% fix to resume stopping at error messages 
 \newcommandtwoopt{\citepads}[3][][]{%
   \nonstopmode%              %% fix to not stop at error message in latex
   \href{http://adsabs.harvard.edu/abs/#3}%
        {\def\hyper@linkstart##1##2{}%
         \let\hyper@linkend\@empty\citep[#1][#2]{#3}}%     %% (Rutten 2000)
   \biblink{#3}{\href{http://adsabs.harvard.edu/abs/#3}{ADS}}%
   \errorstopmode}            %% fix to resume stopping at error messages
 \newcommandtwoopt{\citetads}[3][][]{%
   \nonstopmode%              %% fix to not stop at error message in latex
   \href{http://adsabs.harvard.edu/abs/#3}%
        {\def\hyper@linkstart##1##2{}%
         \let\hyper@linkend\@empty\citet[#1][#2]{#3}}%     %% Rutten (2000)
   \biblink{#3}{\href{http://adsabs.harvard.edu/abs/#3}{ADS}}%
   \errorstopmode}            %% fix to resume stopping at error messages 
 \newcommandtwoopt{\citeyearads}[3][][]{%
   \nonstopmode%              %% fix to not stop at error message in latex
   \href{http://adsabs.harvard.edu/abs/#3}%
        {\def\hyper@linkstart##1##2{}%
         \let\hyper@linkend\@empty\citeyear[#1][#2]{#3}}%  %% 2000
   \biblink{#3}{\href{http://adsabs.harvard.edu/abs/#3}{ADS}}%
   \errorstopmode}            %% fix to resume stopping at error messages 
\makeatother

%% Acronyms
\newacro{ADS}{Astrophysics Data System}
\newacro{NLTE}{non-local thermodynamic equilibrium}
\newacro{NASA}{National Aeronautics and Space Administration}

%% Add popups with meaning to acronyms 
%% NB: only show up in Adobe Reader and do not work with \input or \include
\gdef\acp#1{%
  \pdfmarkupcomment[markup=Underline,color={1 1 1},author={{#1}},opacity=0]%
  {{#1}}{{\acl{#1}}}}

%% Spectral species
\def\MgI{\ion{Mg}{I}}          %% A&A; for aastex use \def\MgI{\ion{Mg}{1}} 
\def\MgII{\ion{Mg}{II}}        %% A&A; for aastex use \def\MgII{\ion{Mg}{2}} 

%% Hyphenation
\hyphenation{Schrij-ver}       %% Dutch ij is a single character


\begin{document}  

\twocolumn[{
\vspace*{4ex}
\begin{center}
  {\Large \bf FYS4150 Project 1: \\1-dimensional Poisson equation}\\[4ex]       
  {\large \bf Marie Foss$^{1}$, 
              Maria Hammerstr{{\o}}m$^{1}$}\\[4ex]
  \begin{minipage}[t]{15cm}
        $^1$ Institute of Theoretical Astrophysics, University of Oslo\\

  {\bf Abstract.}  We discovered \ldots 

  \vspace*{2ex}
  \end{minipage}
\end{center}
}] 



\section{Introduction}\label{sec:introduction}
In this project we will solve the one-dimensional Poissson equation with Dirichlet boundary conditions by rewriting it as a set of linear equations.The equation to be solved is:

\begin{equation}
    \begin{aligned}
	-u''(x) = f(x) \hspace{0.5cm} x\in(0,1), \hspace{0.5cm} u(0) = u(1) = 0
    \end{aligned}
\end{equation}

MORE COMING HERE.


\section{Solving the problem}\label{sec:observations}
\subsection{Simple algorithm}
In our case we are dealing with a simple tridiagonal matrix \textbf{A}:

\begin{equation}
    {\bf A} = \left(\begin{array}{cccccc}
                           2& -1& 0 &\dots   & \dots &0 \\
                           -1 & 2 & -1 &0 &\dots &\dots \\
                           0&-1 &2 & -1 & 0 & \dots \\
                           & \dots   & \dots &\dots   &\dots & \dots \\
                           0&\dots   &  &-1 &2& -1 \\
                           0&\dots    &  & 0  &-1 & 2 \\
                      \end{array} \right)
\end{equation}

\noindent We can therefore rewrite our matrix \textbf{A} in terms of one-dimensional vectors \textit{a, b, c} of length 1:\textit{n}.

MORE COMING HERE.



\subsection{Using a library package}
In addition to solving the linear second-order differential equation Eq. XX using the simple algorithm described above, we want to solve the equation using Gaussian elimination and LU decomposition, then compare the results.

MORE COMING HERE.



\section{Analaysis}    \label{sec:analysis}
The modeling of \dots


\section{Conclusions} \label{sec:conclusions}
The answer is 42.



\begin{acknowledgements}
  We are much indebted to Rob Rutten for exemplary instruction.
  Our research made much use of NASA's Astrophysics Data System.
\end{acknowledgements}



%% references
\bibliographystyle{aa-note} % aa.bst but adding links and notes to references
%%\raggedright              %% only for adsaa with dvips, not for pdflatex
\bibliography{XXX}          %% XXX.bib = your Bibtex entries copied from ADS

\end{document}

